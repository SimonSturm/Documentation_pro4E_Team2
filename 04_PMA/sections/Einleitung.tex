\section{Einleitung}


Im Rahmen des Projekts 4 an der FHNW wird eine Überwachungsanlage für Solarparks entwickelt um verschmutzte oder defekte Solarzellen frühzeitig zu erfassen und zu lokalisieren. Der hierfür genutzte Indikator für die Zellenverschmutzung ist die Modulspannung im Vergleich zur Standardabweichung des gesamten Strings. Um diese Spannungen zu erfassen wird an jeder Solarzelle eine Sensorplatine angebracht, welche direkt von der Zelle (gespeist wird). Der gemessene Wert wird direkt über das Kabel der Solarzellen übertragen und vom Meldegerät empfangen. Dieses vergleicht nun die einzelnen Zellen zueinander und meldet einen Fehler durch eine LED-Lampe direkt auf dem Gerät und zusätzlich durch einen Relay kontakt. Die Bedienung sollte möglichst intuitiv und auch ein nachrüsten in bestehenden Solarparks muss möglich sein.

Die wichtigsten Anforderungen an unsere Überwachungsanlage waren ein niedriger Leistungsverbrauch sowie der schnelle und leichte Einbau an der Solarzelle. Zugleich soll eine Montage ohne zusätzliche Verkabelung und das Nachrüsten in bereits bestehenden Solarparks möglich sein.
 
Um mit dieser sehr umfangreichen Aufgabe zurecht zukommen standen dem Projektteam fünf tatkräftige Fachchoaches zur Seite. Hans Gysin für den Hardwareaufbau, Martin Meier für technische Unterstützung bei Mikrocontroller Problemen, Pascal Schleuniger für Schaltungsfragen, Pascal Buchschacher für das Projektmanagement und Anita Gertiser für die Kommunikation und das Berichtwesen. Um die Arbeit effizienter unter den sieben Teammitgliedern aufteilen zu können wurden drei Gruppen gebildet. Die Hardwarespezialisten kümmerten sich um den Aufbau der Schaltung, während die Softwarespezialisten sich mit der Programmierung des Mikrocontrollers beschäftigten. Die dritte Gruppe unterstützte entweder bei Bedarf bei den einzelnen Teilgebieten oder den Projektleiter.

Zu Beginn wurde das Projekt mit grosser Euphorie begonnen und es wurden schnell erste Fortschritte gemacht. Bis zur Projektwoche verlief alles nach Plan, doch während der Projektwoche sind erste Defizite der Software aufgefallen, welche bis zum Projektende nicht gelöst werden konnten. 

Dieser Bericht beinhaltet sowohl die positiven wie auch die negativen Erfahrungen des Projekts. Die zentralsten und wichtigsten Erfahrungen werden zusätzlich detailliert aufgeführt und erläutert.
