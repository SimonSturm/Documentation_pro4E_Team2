\section{Reflexion}
Im nachfolgenden Kapitel werden vier Ereignisse aus der vorhergehenden Tabelle genauer analysiert und detailliert aufgeführt.

\subsection{Fachbericht}

Der Fachbericht konnte bereits vor dem geplanten Abgabetermin fertig gestellt werden. Dies ist vor allem auf den erhöhten Zeitvorrat zurück zu führen, der entstanden ist, als die Validierung gestrichen werden musste.  Der Fachbericht konnte durch das entstandene Zeitpolster mehrmals an Frau Gertiser abgegeben werden um ein Feedback von ihr einzuholen und dieses auch zeitnah umzusetzen. \newline
Bereits während der Projektwoche wurden drei Versionen der Disposition abgegeben, wodurch die grobe Struktur des Berichtes sehr früh festgelegt werden konnte und aufgrund dessen auch eine detaillierte Arbeitsverteilung möglich war. \newline
Da sich bereits während der Projektwoche abzeichnete, dass das Endprodukt nicht funktionsfähig sein wird, wurde bereits nach der Projektwoche mit dem Fachbericht begonnen um diesen bestmöglich erstellen zu können. \newline
Um eine gute Dokumentation sicherzustellen wurden die Aufgaben in ihre Fachbereiche aufgeteilt und anhand der bereits erstellten Disposition an die einzelnen Teammitglieder verteilt um eine faire Arbeitsverteilung zu gewährleisten. Ebenfalls wurde ein Abgabetermin für den Fachbericht vereinbart um diesen von einer unabhängigen dritten Person korrigieren zu lassen.\newline
Durch die häufigen Feedbacks von Frau Gertiser, aber auch durch teaminternes Korrigieren konnte die Qualität des Fachberichtes deutlich erhöht werden. Ebenfalls konnte durch eine gute Arbeitsverteilung die Überarbeitung einzelner Teammitglieder sowie die Doppelbearbeitung einzelner Dokumententeile vermieden werden. \newline
Damit sich die einzelnen Teammitglieder selbständig ihre Arbeitsaufteilung machen konnten wurde früh mit dem Fachbericht begonnen, was schlussendlich zum gewünschten Ziel geführt hat. Der Fachbericht konnte dadurch ohne grossen Stress vollendet werden und dadurch konnten viele Flüchtigkeitsfehler bereits korrigiert werden.
\newpage

\subsection{Sensorprint}

Der Sensorprint konnte bereits vor dem geplanten Datum bestellt werden, da der Arbeitsaufwand geringer ausfiel als zunächst angenommen. Auch die Software konnte bereits mehrheitlich während der Projektwoche in Betrieb genommen werden. Leider war der Zeitgewinn schlussendlich nicht so gross wie anfänglich angenommen, da noch gewisse Fehler entdeckt wurden.\newline
Die anfängliche Euphorie durch den frühen Erfolg legte sich schnell wieder als gewisse Hardware- und Software-Fehler auftauchten, welche bis zum Projektende teilweise nicht behoben werden konnten. \newline
Die Ursache dieser Abweichung liegt einerseits bei dem guten Engagement während dem Projektstart und andererseits daran, dass der Arbeitsaufwand anfänglich zu hoch eingeschätzt wurde.\newline
Während dieser Zeit hätten die Ressourcen um geplant werden müssen, da der Meldeprint bereits in dieser Phase nicht nach Plan produziert wurde. Durch gewisse Fehler auf dem Sensorprint wurde jedoch davon abgesehen und die Arbeitsverteilung bei belassen. \newline
Im besten Fall hätte durch diese Massnahme die Motivation des Teams gesteigert werden können, denn es hat sich schnell gezeigt, dass der Meldeprint nicht mehr termingerecht fertig gestellt werden kann.\newline
Auch wenn früh erste Erfolge verbucht werden konnten, darf dadurch der Fokus auf das gesamte Projekt nicht vernachlässigt werden. Die anfängliche Euphorie durch erste Teilerfolge hielt nicht sehr lange an und bald war 
\newpage

\subsection{Software des Meldeprints}

Die Software des Meldeprints konnte bis zum Projektende nicht fertiggestellt werden. Dies hat mehrere Ursachen, einerseits fehlte ein detaillierterer Projektstrukturplan der Software, da der Projektleiter weder den Zeitaufwand noch die nötigen Komponenten dieser Software zu Projektbeginn definiert hat. Andererseits war kaum Fachwissen vorhanden und konnte auch während dem Projekt nicht aufgebaut werden. Trotz vielen Stunden mit Unterstützung von Herrn Meier konnte bis am Schluss nur das Menu in Betrieb genommen werden. \newline
Es wurde versucht die verlorene Zeit aufzuholen indem beide freien Projektmitarbeiter der Software zur Seite gestellt wurden. Der Effekt dieser Massnahme hielt sich jedoch stark in Grenzen, da beiden ebenfalls das Fachwissen fehlte und somit nur mehr Zeit in die Software investiert wurde ohne nennenswerte Erfolge. \newline
Die Abweichungen wurden wegen der fehlenden Unterteilung im Projektstrukturplan zu spät entdeckt und es konnten keine Massnahmen mehr ergriffen werden um die Software noch zu retten. Dies hat zur Folge, dass die Validierung ebenfalls nicht mehr vollständig durchgeführt werden konnte.  Die Software wurde aus Zeitgründen nicht mehr fertig gestellt und die freien Ressourcen wurden in die theoretische Erweiterung der Software und den Fachbericht investiert.\newline
Für ein nächstes Projekt ist somit klar, dass ein wichtiger Grundstein für ein erfolgreiches Projekt eine solide und detaillierte Planung notwendig ist und auch die Arbeitspakete genügend klein gewählt werden müssen um die Arbeitsplanung zu kontrollieren und wenn nötig frühzeitig eingreifen zu können.
\newpage 
\subsection{Meldeprint}

Der Meldeprint konnte schlussendlich nur noch teilweise aufgebaut werden, da die grundlegende Planung falsch angesetzt war und ebenfalls gewisse Testaufbauten zu spät erstellt wurden. Um einen reibungslosen Aufbau sicher zustellen hätte der Meldeprint zeitlich versetzt zum Sensorprint aufgebaut werden müssen. \newline
Der Meldeprint wurde mit einer sechs wöchigen Verspätung nur noch teilweise Aufgebaut um einen noch grösseren Zeitverlust zu verhindern. Diese Entscheidung wurde gefällt, da ebenfalls Teile der Software fehlten und die Validierung aus diesem Grund schon gestrichen wurde.  \newline
Die Ursachen für die Zeitverzögerung liegen einerseits beim Projektmanager, da er sich zu wenige Gedanken über die Abfolge der Arbeitspakete gemacht hat, da die gleichen Bauteile bei beiden Prints eingesetzt wurden hätte zuerst der Sensorprint aufgebaut und getestet werden sollen und erst anschliessend das Layout für den Meldeprint erstellt werden. Andererseits ist der Projektmitarbeiter mitten wärend der Realisierung für eine Woche krankheitsbedingt ausgefallen und die restlichen Mitglieder waren mit ihren eigenen Baustellen beschäftigt, dadurch blieb diese Arbeit während einer Woche unerledigt. \newline
Als letzte Massnahme um den Print doch noch zu erstellen, wurden nur noch die Teile des Meldeprints bestückt, welche auch durch die Software angesteuert werden können. Auf die Wunschziele wurde ebenfalls verzichtet. \newline
Die dadurch gewonnenen Zeitreserven wurden wie auch bei der Software in den Fachbericht gesteckt um auch auf Hardwareseite einen möglichst guten Bericht zu erstellen. Trotz aller Verzögerung konnte die Funktion des Meldeprints validiert werden und steht somit für die Präsentation doch noch zur Verfügung.\newline
Um in Zukunft solche Probleme zu umgehen, sollte der Projektleiter seine Planung mit den Mitarbeitern zusammen durchgehen um mögliche Planungsfehler und Engstellen frühzeitig zu erfassen. 
