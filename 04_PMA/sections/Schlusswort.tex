\section{Schlusswort}



Während des Projekts konnten viele wertvolle Erfahrungen gesammelt werden, dabei lässt sich sowohl aus den positiven wie auch aus den negativen Erreignissen viel lernen. So wurde klar, dass der Einsatz und die Motivation der Projektmitglieder im Verlaufe eines Semesters stark variieren kann. Diese Einstellung lässt sich nur schwer beeinflussen, doch sie ist für jedes Projekt von höchster Wichtigkeit. Genügend Freiraum für eigene Ideen und Konzepte kann die Einsatzbereitschaft der Projektmitglieder positiv beeinflussen, jedoch darf dieser Freiraum auch nicht zu gross sein, da sonst die Arbeitsplanung und die Erkennung von Rückständen erschwert werden.

Es hat sich gezeigt, dass besonders der Anfangsphase eines Projektes grosse Beachtung geschenkt werden muss. Bei allen Freiheiten müssen in dieser Phase klare Vorgehensweisen und Ziele vereinbart werden. In einem nächsten Projekt wäre dieser straffen Planung in der Anfangsphase deutlich mehr Rechnung zu tragen.