\section{Einleitung}


Die Klimaerwärmung ist Tatsache! Deshalb steigt die Nachfrage nach alternativen Energielieferanten wie Solaranlagen laufend. Um möglichst effizient Strom zu erzeugen, werden die einzelnen Solarmodule seriell und parallel zu grösseren Solarparks zusammen geschaltet. Jedoch existieren noch viele Probleme im Einsatz. Zu denen gehören unter anderem die Verschmutzung der Oberflächen und defekte Zellen. Diese Störungen bleiben meist lange unentdeckt. Das Problem dabei ist, dass defekte seriell geschaltete Solarmodule auch die funktionierenden behindern, was zu einem Abfall der Gesamtleistung der Anlage führt. Dies kann zu grossen Ertragseinbussen führen.


Im Projekt 4 an der FHNW wird eine Überwachungsanlage für solche Solarparks entwickelt. Eine Sensorplatine soll an jedem Modul befestigt werden, dessen Spannung messen und direkt vom Modul gespiesen werden. Ein zentrales Meldegerät soll ab Netz versorgt werden. Die Sensorplatine und die Meldeplatine sollen ohne weitere Verkabelung kommunizieren. Auch nach dem Aufbau der Solaranlage muss das Überwachungssystem simpel eingebaut werden können. Zudem soll die Bedienung intuitiv sein.

Das Konzept sieht vor, Störungen einzelner Solarpanels zu erkennen, indem die Spannungen an jedem Modul gemessen werden. Solarparks im Umfang von bis zu 50 Solarmodulen pro String und maximal 5 Strings können damit überwacht werden. Das System besteht aus einer zentralen Meldestation für jeden String und einer Sensorplatine, die an jedem einzelnen Solarmodul befestigt wird. Gespeist werden die Sensorplatinen direkt vom Solarmodul. Diese Sensorplatine misst die Ausgangsspannung des Solarmoduls und sendet dies Werte dem Meldegerät. Die Spannungswerte werden über einen AD-Wandler eingelesen, damit sie vom Mikrocontroller weiterverarbeitet werden können. Die Datenübertagung zum Meldegerät geschieht via Powerline, damit keine zusätzliche Verkabelung notwendig ist. Die Meldestation, die ab Netz gespeist wird, empfängt und vergleicht die jeweiligen Spannungen der einzelnen Module mit dem Mittelwert des zugehörigen Strings. Wenn ein Modul stark vom Mittelwert abweicht, wird es vorgemerkt. Bleibt dieser Spannungswert nach der nächsten Messung praktisch unverändert, wird ein Relaiskontakt geschlossen und eine Fehlermeldung auf dem Display angezeigt. Mit dem Relaiskontakt wird eine beliebige, externe Fehlermeldung ermöglicht. Nach der Erstinstallation hat jedes Modul ein eigene Nummer, an der es identifiziert werden kann.
