\section{Einleitung}


Wegen der momentan schon zu beobachtenden Auswirkungen der Klimaerwärmung, sind Alternativen zum Erdöl und anderen fossilen Brennstoffen gesucht. Erneuerbare Energiegewinnungsmethoden erleben einen Aufschwung. Auch Photovoltaik-Anlagen verbreiten sich mit zunehmendem Trend. Neben Wind und Wasser ist die Sonne eine der wichtigsten alternativen Energielieferanten. Um möglichst effizient Strom zu erzeugen, werden die einzelnen Solarzellen seriell und parallel zu grösseren Solarparks zusammen geschaltet. Jedoch existieren noch viele Probleme im Einsatz. Zu denen gehören unter anderem die Verschmutzung der Oberflächen und defekte Zellen. Diese Störungen bleiben meist lange unentdeckt. Das Problem dabei ist, dass nicht funktionierende seriell geschaltete Solarmodule auch die funktionierenden behindern. Die Gesamtleistung der Anlage wird begrenzt. Dies kann zu grossen Ertragseinbussen führen.


Im Projekt 4 wird eine Überwachungsanlage für solche Solarparks entwickelt. Das Konzept sieh vor,Störungen zu erkennen, indem die Spannungen an jedem PV-Modul gemessen werden. Solarparks im Umfang von bis zu 50 Solarmodulen pro String und maximal 5 Strings können damit überwacht werden. Das System besteht aus einer zentralen Meldestation für jeden String und einer Sensorplatine, welche an jedem einzelnen Solarpanel montiert wird. Gespeist werden die Sensorplatinen direkt von Solarmodul. Sie misst die dort anliegende Spannung und sendet diese dem Meldegerät. Die Spannungswerte werden über einen AD-Wandler eingelesen, damit sie vom Mikrocontroller weiterverarbeitet werden können. Die Datenübertagung zum Meldegerät geschieht via Powerline, da keine zusätzlichen Leitungen angebracht werden. Mittels eines Powerline Tranceivers werden die Daten übermittelt. Die Meldestation, welche ab Netz gespeist wird, empfängt und vergleicht die jeweiligen Spannungen der einzelnen Module mit dem Mittelwert des zugehörigen Strings. Wenn ein Modul stark vom Mittelwert abweicht, steht es unter Beobachtung. Bleibt nach dieser Spannungswert nach nach der nächsten Messung praktisch unverändert,  wird ein Relaiskontakt geschlossen und eine Fehlermeldung auf einem Display angezeigt. Mit Relais-Kontakt wird eine beliebige, externe Fehlermeldung ermöglicht. Nach der Erstinstallation hat jedes Modul ein eigene Nummer, an der es identifiziert werden kann. Anhand eines Registers, in welchem alle Spannungswerte eingetragen sind, wird bekannt bei welcher Zelle das Problem entsteht.
