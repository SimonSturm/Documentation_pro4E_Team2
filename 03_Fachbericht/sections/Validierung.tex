\section{Validierung}
In den nachfolgenden Kapiteln findet sich eine Zusammenfassung aller wichtigen Erkenntnisse sowie eine Zusammenstellung der erzielten Messergebnisse. 
\subsection{Spannungsmessung}
Als Indiz für eine defekte oder verdunkelte Solarzelle wird die Zellenspannung gemessen, daher ist bei dieser die Messgenauigkeit sehr wichtig und wurde vom uftraggeber auf $\pm$0.1V festgelegt. Um dies zu überprüfen wird die angelegte Spannung mit einem weiteren Messgerät aufgezeichnet und mit dem ermittelten Wert verglichen. Die berechnete Abweichung beträgt $\pm$??V, daher sollte die tatsächliche Messungenauigkeit zwischen $\pm$??V und $\pm$0.1V liegen. In diesem Kapitel werden die Ergebnisse der Genauigkeitsmessung aufgeführt.
\subsection{Datenübertragung via Powerline}
Das Ziel dieser Messung ist es, eine vorher definierte Anzahl Bits via Powerline zu übertragen. Dafür werden an einem spannungsführenden Kabel zwei Ferritkerne, mit einem gewissen Abstand dazwischen, angebracht und über eine Spule mit dem jeweiligen Transceiver verbunden. Das vom ersten Transceiver gesendete Bitmuster sollte beim zweiten Transceiver empfangen werden und das selbe Bitmuster wird wieder ausgeben. Die Ergebnisse dieser Messung werden im folgenden Kapitel erläutert.
\subsection{Fehlererkennung eines Moduls}
Damit eine einfache Lokalisierung des defekten Moduls sichergestellt ist, hat jeder Sensorprint eine eigene Erkennungsnummer. Um sicher zu stellen, dass diese Erkennungsnummer korrekt übertragen und verarbeitet wird, ist ein einfaches Testkonzept vorgesehen. Das Sensorboard wird auf eine gewisse Erkennungsnummer eingestellt und der Meldeprint kalibriert. Anschliessend wird die Nummer geändert was zur Folge hat, dass der Meldeprint einen Fehler ausgibt und diesen auf dem LCD-Display mit der dazugehörigen Erkennungsnummer anzeigt. Im folgendem Kapitel sind die Ergebnisse dieses Tests aufgeführt.
\subsection{Kollisionserkennung}
Alle Mikrocontroller senden ihre Daten in einem vorher festgelegten Zeitintervall, dadurch kommt es zu Kollisionen. Um sicherzustellen, dass der Meldeprint nur mit korrekt übertragenen Messwerten rechnet, werden zusätzliche Informationen gesendet. Überprüft wird diese CRC genannte Methode indem die Übertragung vorzeitig unterbrochen wird. Der Mikrocontroller sollte damit in der Lage sein, das fehlerhafte Packet zu erkennen und dieses zu ignorieren. Die Resultate dieser Messung sind in diesem Kapitel aufgeführt.