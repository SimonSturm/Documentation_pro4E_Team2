\section{Validierung}
Im folgenden Kapitel findet sich eine Zusammenfassung aller wichtigen Erkenntnisse sowie eine Zusammenstellung der erzielten Messergebnisse. 
\subsection{Spannungsmessung}
Als Indiz für eine defekte oder verdunkelte Solarzelle wird die Zellenspannung gemessen, daher ist bei dieser die Messgenauigkeit sehr wichtig und wurde auf 0.1V festgelegt. Um dies zu überprüfen wird die angelegte Spannung mit einem weiteren Messgerät aufgezeichnet und mit dem ermittelten Wert verglichen. Im nachfolgenden Kapitel werden die Ergebnisse der Genauigkeitsmessung aufgeführt.
\subsection{Datenübertragung via Powerline}
Das Ziel dieser Messung ist es, eine vorher definierte Anzahl Bits via Powerline zu übertragen. Dafür werden an einem spannungsführenden Kabel zwei Ferritkerne, mit einem gewissen Abstand dazwischen, angebracht und über eine Spule mit dem jeweiligen Transceiver verbunden. Die Ergebnisse dieser Messung werden im nächsten Kapitel erläutert.
\subsection{Fehlererkennung eines Moduls?}
Damit eine einfache Lokalisierung des defekten Moduls sichergestellt ist, hat jeder Sensorprint eine eigene Erkennungsnummer. Um sicher zu stellen, dass diese Erkennungsnummer korrekt übertragen und verarbeitet wird, ist ein einfaches Testkonzept vorgesehen. Das Sensorboard wird auf eine gewisse Erkennungsnummer eingestellt und der Meldeprint kalibriert. Anschliessend wird die Nummer geändert was nun zur Folge hat, dass der Meldeprint einen Fehler ausgibt und diesen auf dem LCD-Display anzeigt. Im folgendem Kapitel sind die Ergebnisse dieses Tests aufgeführt.
\subsection{Kollisionserkennung}
Alle Mikrocontroller senden ihre Daten in einem vorher festgelegten Zeitintervall, dadurch kommt es zu Kollisionen. Um sicherzustellen, dass der Meldeprint nur mit korrekt übertragenen Messwerten rechnet, werden zusätzliche Informationen gesendet. Überprüft wird diese CRC genannte Methode indem die Übertragung vorzeitig unterbrochen wird. Die Resultate dieser Messung sind im nächsten Kapitel aufgeführt.