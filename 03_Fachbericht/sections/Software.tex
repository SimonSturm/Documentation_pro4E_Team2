\section{Software}
Die Software bildet den wichtigsten Teil des Bindeglieds zwischen Benutzer und Hardware. Als erstes muss den Sensorplatinen eine Indentifikationsnummer zugeteilt und in der Zentrale die Anzahl Module angegeben werden. Die Gemessenen Spannungswerte der Solarmodule werden via Powerline an die Meldeplatine gesendet. Es sollen keine extra Datenkabel verwendet werden. Diese Daten müssen in der Meldezentrale verarbeitet werden. Eine allfällige Abweichung des Spannungswertes bei einem oder mehreren Modulen sollen erkannt und identifiziert werden. Worin genau diese Abweichung besteht, wird später erläutert. Nach der Identifikation muss ein solcher Fehler gemeldet werden. Die Kommunikation zwischen Hardware und Software basiert auf SPI. Das Endprodukt setzt aus dem Melde- und Sensor-Print zusammen. In dieser Unterteilung werden in den nächsten Kapiteln alle Softwarekomponenten beschrieben.
\subsection{Sensorprint}
Der Sensorprint übernimmt die Spannungsmessung und sendet diese mit der Identifikationsnummer via Powerline an den Meldeprint. Die Kopplung in die Powerline gelingt mit einem Transceiver. In den Folgenden Kapiteln wird die Software weiter unterteilt und jeder Bereich einzeln erläutert.
\subsubsection{Aufbau und Abläufe (Flussdiagramm)}
\subsubsection{Inbetriebnahme}
wieso-was-wie-unter welchen Bedingungen
\subsubsection{Spannungsmessung}
wieso-was-wie-unter welchen Bedingungen
\subsubsection{Identifikation}
wieso-was-wie-unter welchen Bedingungen
\subsubsection{Transceiveransteuerung}
wieso-was-wie-unter welchen Bedingungen
\subsection{Meldeprint}
Der Meldeprint empfängt die Daten des Sensorprints mit Hilfe eines Receivers. Die Spannungswerte werden in Verbindung mit der Identifikationsnummer gespeichert und mit dem Sollwert vergleicht. Der Sollwert entspricht dem arithmetischen Mittelwert der Spannungen dieses einen Strings von Modulen. Bei einer Abweichung von *** Prozent, wird dasjenige Modul vorgemerkt. Falls nach einer weiteren Messung die Spannung des selben Moduls gleich vom Mittelwert abweicht, muss ein Fehler gemeldet werden. Die Identifikationsnummer wird auf dem Display angezeigt. Die in Standard C geschriebene Software bezieht sich auf den ATMEGA328P von Atmel. Das Hauptprogramm beinhaltet die Initialisierungen der Pins, des LCD Display, des Transceivers und des Drehgebers.Die einzelnen Bereiche werden unten genauer erläutert.
\subsubsection{Aufbau und Abläufe (Flussdiagramm)}
\subsubsection{Inbetriebnahme}
wieso-was-wie-unter welchen Bedingungen
\subsubsection{Receiver}
Wie werden die Daten empfangen und wie werden sie gespeichert.
wieso-was-wie-unter welchen Bedingungen
\subsubsection{Fehlererkennung}
Aus Zeitgründen konnte die Datenverarbeitung noch nicht programmiert werden. Der Plan ist aber die gemittelten Spannungswerte, die jede Sensorplatine liefert, nochmals über alle Module des Strings zu mitteln. Damit kann dann die Standardabweichung berechnet werden. Ist die Abweichung eines Wertes mehr als 20 Prozent über der Standardabweichung, wird diese Solarmodul vorgemerkt. Diese Prozedere wiederholt sich stündlich. Weicht der Spannungswert des vorgemerkten Moduls nach der nächsten Stunde noch immer so stark vom Mittelwert ab, wird eine Fehlermeldung losgeschickt.

wieso-was-wie-unter welchen Bedingungen
\subsubsection{Fehlermeldung (Display und Relais)}
Die Fehlermeldung besteht aus der Fehlerankündigung selbst und der Identifikationsnummer der Sensorplatine, welche am defekten oder verschmutzten Modul befestigt ist. Die Fehlerankündigung ist als Text wie beispielsweise "Problem beim Modul..." auf dem Display zu sehen. Die Identifikationsnummer des Moduls wird von einer 8bit Binärzahl in eine Dezimalzahl umgewandelt und so angezeigt.

wieso-was-wie-unter welchen Bedingungen
\subsubsection{Benutzerinterface (Display und Drehgeber)}
wieso-was-wie-unter welchen Bedingungen



