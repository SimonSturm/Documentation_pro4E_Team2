\section{Schlusswort}
% 10 volle Zeilen sind circa eine halbe A4 Seite
Das Solar Management System besitzt eine einfach Aufgabe, die Überwachung einer Solaranlage. Die Installation sowie Bedienung und Überwachung sind ohne grosse Vorkenntnisse zu handhaben. Die Installation ist darum einfach, weil der Sensorprint nur auf das Modul aufgebracht werden muss mit einer Verbindung zu den beiden DC-Leitungen. Der Meldeprint wird analog beim Wechselrichter angebracht. Das System ist damit ohne grossen Aufwand erweiter- und austauschbar. Ist der Meldeprint einmal mit Energie versorgt, beginnt ''SMS'' selbstständig die Modulspannungen zu messen. Für eine individuelle Benachrichtigung hat der Benutzer ein Relaiskontakt zu seiner Verfügung.\\
Das Projekt konnte nicht vollständig fertiggestellt sowie umfangreich getestet werden. Ideen, Ansätze und Pläne sind vorhanden, jedoch fehlen dem Team Spezialisten für die Umsetzung. Die Arbeiten sind bei der Realisierung stehen geblieben. Die Hardware beider Prints ist beinahe fertig, im Gegensatz zur Software, die nicht vollständig geschrieben wurde. Besonders der mahematische Teil beim Meldeprint ist wenig ausgeprägt. Dafür konnte das Benutzerinterface realisiert werden mit der Anzeige auf dem Display. Der Sensorprint ist sehr weit fortgeschritten, benötigt jedoch noch einige Codezeilen für eine korrekte Übertragung auf der Powerline. \todo{Was fehlt noch genau am Sensorprint?} Die Validierung konnte wegen den oben genannten Punkten nicht gestartet werden, es wurde jedoch ein theoretisches Konzept erarbeitet.