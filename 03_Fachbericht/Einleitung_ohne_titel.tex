\section{Einleitung}


Photovoltaik-Anlagen verbreiten sich mit zunehmendem Trend. In Zeiten wie diesen, in denen man immer mehr auf fossile Energieträger verzichten will, besonders. Neben Wind und Wasser ist die Sonne eine der wichtigsten alternativen Energielieferanten. Um möglichst effizient Strom zu erzeugen, werden die einzelnen Solarzellen seriell und parallel zu grösseren Solarparks zusammen geschaltet. Jedoch existieren noch viele Probleme im Einsatz. Zu denen gehören unter anderem die Verschmutzung der Oberflächen und defekte Zellen. Diese Störungen bleiben meist lange unentdeckt, was starke Auswirkungen auf die gesamte Leistung der Anlage hat. Dies kann zu grossen Energie- und Ertragseinbussen führen.


Im Projekt 4 wird eine Überwachungsanlage für solche Solarparks entwickelt. Eine Sensorplatine, welche an jedem einzelnen Solarmodul montiert wird, gehört zur Ausstattung. Diese sendet Information zu dem jeweiligen Modul an die zentrale Meldestation. Die externe Meldestation wertet die Daten aus und entscheidet, ob alles in Ordnung ist oder eine Fehlermeldung losgeschickt werden muss. Nebst der eigentlichen Fehlermeldung, muss bekannt sein, bei welcher Zelle das Problem entsteht.



Die Idee besteht darin, mithilfe von Spannungsmessungen, Störungen zu erkennen. Die am Solarpanel angebrachte Sensorplatine soll  Spannungswerte vom Panel bekommen. Diese werden über einen AD-Wandler eingelesen, damit sie vom Mikrocontroller weiterverarbeitet werden können. Mittels eines Powerline Tranceivers werden die Daten übermittelt, so werden keine zusätzlichen Leitungen benötigt. Aus den aufgezeichneten Spannungswerten wird auf der Kontrollplatine ein Durchschnittswert berechnet und die einzelnen Werte damit verglichen. Ein Panel mit zu stark abweichenden Werten wird weiter beobachtet und falls diese Abweichtung bis zur nächsten Messung bestehen bleibt, wird ein Relaiskontakt geschlossen und eine Fehlermeldung auf einem Display angezeigt. Für die Kennung des defekten oder verschmutzten Solarmoduls  werden die einmaligen Identifikationsnummern der Mikrocontroller auf der Sensorplatine verwentet.
