\documentclass[a4paper]{fhnwreport} %Legt grundlegende Formatierungen wie Schriftarten, Ort Seitenzahlen etc. fest.

\begin{document}
\section{Einleitung}

\subsection{Ausgangslage}

Photovoltaik-Anlagen verbreiten sich mit zunehmendem Trend. In Zeiten wie diesen, in denen man immer mehr auf fossile Energieträger verzichten will, besonders. Neben Wind und Wasser ist die Sonne eine der wichtigsten alternativen Energielieferanten. Um möglichst effizient Strom zu erzeugen, werden die einzelnen Solarzellen seriell und parallel zu grösseren Solarparks zusammen geschaltet. Jedoch existieren noch viele Probleme im Einsatz. Zu denen gehören unter anderem die Verschmutzung der Oberflächen und defekte Zellen. Diese Störungen bleiben meist lange unentdeckt, was starke Auswirkung auf die gesamte Leistung der Anlage hat. Dies kann zu grossen Energie- und Ertragseinbussen führen.

\subsection{Ziel}

Im Projekt 4 wird eine Überwachungsanlage für solche Solarparks entwickelt. Eine Sensorplatine, welche an jedem einzelnen Solarmodul montiert wird, gehört zur Ausstattung. Diese sendet Information zu dem jeweiligen Modul an die zentrale Meldestation. Die externe Meldestation wertet die Daten aus und entscheidet, ob alles in Ordnung ist oder ob eine Fehlermeldung losgeschickt werden muss. Nebst der eigentlichen Fehlermeldung, muss bekannt sein, bei welcher Zelle das Probleme entsteht.

Idee

Ergebnisse
\end{document}