%\documentclass[a4paper]{fhnwreport} %Legt grundlegende Formatierungen wie Schriftarten, Ort Seitenzahlen etc. fest.
%
%\graphicspath{{./graphics/}}%Change according to graphics folder!
%
%\begin{document}
\section{Anforderungen}

\subsection{Sensorplattine}
\begin{enumerate}
\item Der Einbau in die Anschlussbox jedes PV-Moduls ist möglich.
\item Die Versorgung geschieht direkt vom Modul (ohne Pufferung).
\item Mit Strom bis 10 A wird gearbeitet.
\item	Die Spannungsmessung erfolgt, mit einer Genauigkeit von ca. 0.1 V, im Bereich zwischen 12 V bis 60 V.
\item	Die Modulspannung wird über die Powerleitung an das zentrale Meldegerät gesendet.
\end{enumerate}

\subsection{Meldegerät}
\begin{enumerate}
\item	Der Einbau in den Schaltschrank ist möglich.
\item	Ein Gehäuse ist vorhanden.
\item	Die Versorgung erfolgt ab Netz (230 V/ 50 Hz).
\item	Das Gerät empfängt und vergleicht die gemeldeten Spannungen der einzelnen Module.
\item	Durch Abweichung der Spannung wird für oder gegen eine Fehlermeldung (über Relaiskontakt) entschieden.
\item	Im Falle eines Defektes wird das fehlerhafte Modul ermittelt und angezeigt (LED).
\end{enumerate}

\subsection{Wunschziele}
\begin{enumerate}
\item	Fehler- und Statusmeldungen können über SMS versendet werden.
\item	Auch der Strom in jedem Modul wird gemessen und gemeldet.
\item	Die Temperatur in jedem Modul wird gemessen und gemeldet.
\end{enumerate}
%\end{document}